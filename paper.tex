\documentclass{IEEEtran}

\begin{document}
\title{Behavioral Analysis of the Agent-Based Community Grid Solution for the Large Hadron Collider beauty Experiment}
\author{Daniela Remenska \and Tim A.C. Willemse \and Henri Bal \and 
Kees Verstoep \and Wan Fokkink \and Jeff Templon}

\maketitle

\begin{abstract}
DIRAC (Distributed Infrastructure with Remote Agent Control) is the Grid
solution designed to support production activities as well as user data
analysis for the Large Hadron Collider beauty experiment. It consists of
cooperating distributed services and a plethora of light-weight agents
delivering the workload to the Grid resources.  Services accept requests
from agents and running jobs, while agents actively fulfill specific
goals. Services maintain database back-ends to store dynamic state
information of entities such as jobs, queues, or staging requests. Agents
continuously check for changes in the service states, and react to these
accordingly. The logic of each agent is rather simple; the main source
of complexity lies in their cooperation. These agents run concurrently,
and communicate using the services' databases as a shared memory for
synchronizing the state transitions. Despite the effort invested in
making DIRAC reliable, entities occasionally get into inconsistent
states. Tracing and fixing such behaviors is difficult, given the
inherent parallelism among the distributed components, and the size of
the implementation.

In this paper we present an analysis of DIRAC with mCRL2, process
algebra with data. We have reverse engineered two critical and related
DIRAC subsystems, and subsequently modeled their behavior with the mCRL2
toolset. This enabled us to easily locate race conditions and livelocks
which were confirmed to occur in the real system. We further formalized
and verified several behavioral properties of the two modeled subsystems.

\textbf{Keywords:} DIRAC, service-oriented architecture, agents, stager,
mCRL2,  model checking,  process algebra

\end{abstract}


\section{Introduction}

The Large Hadron Collider beauty (LHCb) experiment \cite{LHCb} is one of
the four large experiments conducted on the Large Hadron Collider (LHC)
accelerator, built by the European Organization for Nuclear Research
(CERN). Immense amounts of data are produced at the LHC accelerator, and
subsequently processed by physics groups and individuals worldwide. The
sheer size of the experiment is the motivation behind the adoption of
the Grid computing paradigm. The Grid storage and computing resources
for the LHCb experiment are distributed across several institutes in
Europe. To cope with the complexity of processing the vast amount of
data, a complete Grid solution, called DIRAC (Distributed Infrastructure
with Remote Agent Control) \cite{Tsa+:07,ST:07}, has been designed and
developed for the LHCb community.

DIRAC forms a layer between the LHCb community of users and the
heterogeneous Grid resources, in order to allow for optimal and reliable
usage of these resources. It consists of many cooperating distributed
services and agents which deliver workload to the resources. In
particular, services are passive modules which accept incoming requests
from agents and running jobs. Each service has an associated database
back-end to store dynamic state information of entities such as jobs,
queues, or staging requests. Agents are light-weight independent
components that fulfill specific system functions. Using a polling
strategy they continuously observe for updates in the service states
and react accordingly. The logic of each individual agent is relatively
simple; the overall system complexity emerges from the cooperation among
them. Namely, these agents run concurrently, and communicate using
the services' databases as a shared memory (blackboard paradigm \cite{McMB:96})
for synchronizing the entities’ state transitions.  

Although much effort is invested in making DIRAC reliable, entities
occasionally get into inconsistent states, leading to a potential loss
of efficiency in both resource usage and manpower. Debugging and fixing
the root of such encountered behaviors becomes a formidable mission due
to multiple factors: the inherent parallelism present among the system
components which are deployed on different physical machines, the size
of the implementation (around 150000 lines of Python code), and the
distributed knowledge of different subsystems within the collaboration.

In this paper we propose the use of more rigorous (formal) methods
for improving software quality. Model checking \cite{GW:05} is one such
technique for analysis of an abstract model of a system, and verification
of certain system properties of interest. Unlike conventional testing,
it allows full control over the execution of parallel processes and also
supports fully automated exhaustive state-space exploration.

We used the mCRL2 language \cite{GMR+} and toolset \cite{Groote:08}
to model the behavior of two critical and related DIRAC components: the
workload management (WMS) and the storage management system (SMS). Based
on Algebra of Communicating Processes (ACP) \cite{BBR:09}, mCRL2 is able
to deal with generic data types as well as user-defined functions for
data transformation. This makes it particularly suitable for modeling
the data manipulations made by DIRAC's agents. Visualizing the state
space and replaying scenarios with the toolkit's simulator enabled us
to gain insight into the system behavior, incrementally improve the
model, and to already detect critical race-conditions and livelocks,
which were confirmed to occur in the real system. Some of them were a
result of simple coding bugs; others unveiled more elementary design
problems. We further formulated, formalized and verified several general
and application-specific properties.

The idea of modeling existing systems using
formal techniques is as such not new. Earlier studies
(\cite{HJ:04,Pa:06,Hol:90,Fok:11,BR:01,BFGPP:05,ISK:06}) mostly focused
on modeling and verifying hardware or communication protocols, since
the formal languages and tools at hand were not sufficiently mature
to cope with more complex data-intensive distributed systems. More
recently, success stories on modeling real-life concurrent systems with
data have been reported (\cite{HKW:11,BMU:09,HMS:11,VM:05,Plo:09}).
In \cite{HMS:11} the authors have implemented a tool for automatic
translation of the SystemC language into mCRL2 statements. This greatly
simplifies the analysis, but has so far been feasible only when the
language of implementation is domain-specific, or alternatively, a
reasonably small subset of a general-purpose language is considered for
translation. The only exception in this respect is the Java Pathfinder
tool \cite{VM:05} used to find deadlocks and other behavioral properties
in Java software systems developed by NASA.

We believe that the challenges and results of this work are unique in
a number of aspects. First, to the best of our knowledge, the code-base
and the number of concurrent components engaged in providing DIRAC’s
functionality considerably outnumber previous industrial cases. Second,
the choice of Python as implementation platform has lead to prevailing
usage of dynamic structures (whose types and sizes are determined at
runtime) throughout DIRAC, challenging the transition to an abstract
formal representation. We have nevertheless established general
guidelines on extracting a model outline from the implementation. Third,
analysis of this kind is typically performed after a problem has already
surfaced in the real system, as a means to understand the events which
lead to it and test for possible solutions. We managed to stumble on an
actual bug at the same time it was observed in practice, which increased
our confidence in the soundness of the model.

The paper is organized as follows. Section 2 introduces the architecture
of DIRAC, focusing on the two studied subsystems. Section 3 gives a brief
overview of the mCRL2 language, and describes our approach to abstracting
and modeling the behavior of these subsystems. Section 4 presents the
analysis with the mCRL2 toolset and the issues detected. Section 5
concludes and discusses future work. 

\section{DIRAC: A Community Grid Solution}


The development of DIRAC started in 2002 as a system for production
of simulation (Monte Carlo) data that would serve to verify theory,
aspects of the LHCb detector design, as well as to optimize algorithms. It
gradually evolved into a complete community Grid solution for data and job
management, based on a general-purpose framework that can be reused by
other communities besides LHCb. Over the years it has been considerably
reengineered in order to meet the requirements for processing the real
data coming from the LHC. Today, it covers all major LHCb tasks starting
with the raw data transfer from the experiment’s detector to the
Grid storage, several steps of data processing, up to the final user
analysis. Python was chosen as the implementation language of DIRAC,
since it enables rapid prototyping and development of new features,
as well as quickly fixing encountered problems.

DIRAC follows the Service Oriented Architecture (SOA) paradigm,
accompanied by a network of lightweight distributed agents which animate
the system. Its main components are depicted in Figure .

The services are passive components which react to requests from their
clients, possibly soliciting other services in order to fulfill the
requests. They run as permanent processes deployed on a number of
high-availability hosts (VO-boxes) at CERN, and persist the dynamic
system state information in database repositories. To provide a certain
level of load balancing and redundancy, multiple mirror instances of each
service can be deployed on different machines. The user interfaces, agents
or running jobs can act as clients placing the requests to DIRAC’s
services. All clients and services are built in the DISET framework [21]
which provides secure access and flexible authorization rules.

Agents are active components that fulfill a limited number of specific
system functions. They are built around a common framework which provides
a uniform way for configuration, deployment, control, and logging of each
agent activity.  They can run in different environments, depending on the
nature of their mission. Some are deployed close to the corresponding
services, while others run on the Grid worker nodes.  Examples of the
later are the so-called Pilot Agents which are part of the Workload
Management System, explained in more details in the following section
Workload Management System. All DIRAC agents repeat the same logic in
each iteration cycle: they observe for changes in the service states,
and react accordingly by initiating actions (like job submission or data
transfer) which may update the states of various system entities.

Resources are software abstractions of the underlying heterogeneous
Grid computing and storage resources allocated to LHCb, providing a
uniform interface for access to them. The resources are controlled by
the site managers and made available through middleware services such
as gLite~\cite{XXX}.

The DIRAC functionality is exposed to users and developers through a
rich set of command-line tools forming the DIRAC API, complemented by a
Web portal for visual monitoring the system behavior and controlling the
ongoing tasks. Both the Web and command-line interfaces ensure secure
system access using X509 certificates.

\begin{itemize}
\item Explain why you’ve chosen these two. 
\begin{enumerate} 
\item They are related. Stager is crucial for production activities. 
\item Workload is the driving force
\end{enumerate}
\item Explain briefly the other subsystems’ functionality after these two
\end{itemize}

\bibliographystyle{IEEEtran}
\bibliography{bibliography}


\end{document}
